\documentclass[a4paper]{article}
\usepackage[T1]{fontenc}
\usepackage[utf8]{inputenc}
\usepackage[italian]{babel}
\begin{document}
\section*{Relazione Smart Coffee Machine} 
\subsection*{Introduzione}
Lo scopo del progetto era quello di realizzare, tramite la piattaforma Arduino e la comunicazione seriale con una piattaforma realizzata in Java, un simulatore di una macchina del caffè intelligente, che con la rilevazione di movimento e distanza capisca quando accendersi, prepararsi per fare una caffè e tornare in modalità stand by.
\subsection*{Architettura e Progettazione Iniziale}
Siamo partiti ragionando che avremmo avuto più task, ognuno dei quali eseguito in modo periodico da uno scheduler e che avrebbero comunicato attraverso l'uso di variabili condivise.\\ \\ \\
\textbf{Questi sono i seguenti task:}
\begin{itemize}
\item {MovementTask}:


\item {DistanceTask}:



\item {Maintenance Task}:


\item {MakeCoffee Task}:



\item {Main Task}:







\end{itemize}
\newpage
\null
\vfill
Enrico Gnagnarella, Anis Lico, Tommaso Ghini
\clearpage

\end{document}